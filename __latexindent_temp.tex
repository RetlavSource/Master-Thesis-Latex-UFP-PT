 \documentclass[oneside,12pt]{Classes/UFP}

%% Defina as propriedades do PDF gerado
\ifpdf
    \pdfinfo { /Title  (UFP THESIS)
               /Creator (TeX)
               /Producer (pdfTeX)
               /Author (Christophe Soares csoares@ufp.edu.pt)
               /CreationDate (D:20120101000000)  %format D:YYYYMMDDhhmmss
               /ModDate (D:20121101120000)
               /Subject (Como escrever uma tese em LaTeX)
               /Keywords (PhD, Thesis)}
    \pdfcatalog { /PageMode (/UseOutlines)
                  /OpenAction (fitbh)  }
\fi

% Titulo

\title{Template em \LaTeXe para a \\ [1ex] % espaçamento
	Universidade Fernando Pessoa
}

\ifpdf
  \author{\href{mailto:csoares@ufp.edu.pt}{Christophe Soares}}
  \collegeordept{\href{http://fct.ufp.pt/t}{Faculdade de Ciências e Tecnologia}}
  \university{\href{http://www.ufp.pt}{Universidade Fernando Pessoa}}
% insert below the file name that contains the crest in-place of 'UnivShield'
  \crest{\includegraphics[width=25mm]{UFP}}
\fi

\degree{Doctor of Philosophy / Master of Science}
\degreedate{2017}

% turn of those nasty overfull and underfull hboxes
\hbadness=10000
\hfuzz=50pt

% Put all the style files you want in the directory StyleFiles and usepackage like this:
\usepackage{StyleFiles/watermark}

% espaçamento de um e meio entre linhas
\onehalfspacing

\begin{document}

% defina a lingua do documento
%\language{english}


% Documento em fase de edição (comentar ou descomentar consoante precisar
\watermark{DRAFT}


\maketitle

%set the number of sectioning levels that get number and appear in the contents
\setcounter{secnumdepth}{3}
\setcounter{tocdepth}{3}

\frontmatter % book mode only
\pagenumbering{roman}

% empty page
\newpage
\thispagestyle{empty}
\mbox{}

\include{Chapters/Abstract/abstract}

% Thesis Abstract -----------------------------------------------------


%\begin{abstractslong}    %uncommenting this line, gives a different abstract heading
\begin{abstractsEN}        %this creates the heading for the abstract page

Write your summary here...

\end{abstractsEN}
%\end{abstractlongs}




% Thesis Abstract -----------------------------------------------------


% \begin{abstractslong}    %uncommenting this line, gives a different abstract heading
\begin{abstractsFR}        %this creates the heading for the abstract page

Écrivez votre résumé ici...


\end{abstractsFR}
% \end{abstractlongs}



\include{Chapters/Dedication/dedication}
% Thesis Acknowledgements ------------------------------------------------


\begin{acknowledgements}      %this creates the heading for the acknowlegments


Gostaria de agradecer...

\end{acknowledgements}




\renewcommand{\contentsname}{Índice}
\tableofcontents

\renewcommand{\listfigurename}{Índice de Figuras}
\listoffigures

\renewcommand{\listtablename}{Índice de Tabelas}
\listoftables

\renewcommand{\nomname}{Lista de Acrónimos}
\printnomenclature  %% Print the nomenclature
\addcontentsline{toc}{chapter}{Nomenclatura}


\renewcommand{\chaptername}{Cap{\'i}tulo}

\mainmatter % book mode only
\include{Chapters/Introduction/introduction}
% \pagebreak[4]
% \hspace*{1cm}
% \pagebreak[4]
% \hspace*{1cm}
% \pagebreak[4]

\chapter{Segundo Capítulo}
\graphicspath{{Chapters/Chapter1/Chapter1Figs/PNG/}{Chapters/Chapter1/Chapter1Figs/PDF/}{Chapters/Chapter1/Chapter1Figs/}}

\section{Primeiro Parágrafo}
Agora começo com o meu primeiro parágrafo aqui... com uma nota de rodapé \footnote{nota de rodapé}.


Segue-se uma equação :
\begin{eqnarray}
CIF: \hspace*{5mm}F_0^j(a) &=& \frac{1}{2\pi \iota} \oint_{\gamma} \frac{F_0^j(z)}{z - a} dz
\end{eqnarray}


\section{Segundo Parágrafo}
e aqui posso escrever mais ... \cite{texbook}

\subsection{Primeiro sub-parágrafo}
... e mais ...
agora posso citar individualmente : \cite{latex} e \cite{texbook}
e \cite{Rud73}; ou agrupando : \cite{latex, texbook, Rud73}.

Vou agora incluir uma imagem :
\begin{figure}[!htbp]
  \begin{center}
    \leavevmode
    \ifpdf
      \includegraphics[height=6in]{aflow}
    \fi
    \caption{Legenda}
    \label{FigAir}
  \end{center}
\end{figure}

Posso colocar uma referência cruzada no texto desta forma (\ref{FigAir}) e esta referência cruzada encontra-se na página \pageref{FigAir}. 


\chapter{Terceiro Capítulo}

\graphicspath{{Chapters/Chapter2/Chapter2Figs/PNG/}{Chapters/Chapter2/Chapter2Figs/PDF/}{Chapters/Chapter2/Chapter2Figs/}}

\section{Primeiro Parágrafo}
Segue-se uma tabela:
\begin{table}[tbh!]
\caption{Tabela} 
\label{tab:demo-1}
\centering
\begin{tabular}{l*{6}{c}r}
\hline
Team              & P & W & D & L & F  & A & Pts \\
\hline
FC Porto & 6 & 4 & 0 & 2 & 10 & 5 & 12  \\
Celtic            & 6 & 3 & 0 & 3 &  8 & 9 &  9  \\
FC Copenhagen           & 6 & 2 & 1 & 3 &  7 & 8 &  7  \\
SL Benfica     & 6 & 2 & 1 & 3 &  5 & 8 &  7  \\
\end{tabular}
\end{table}

\chapter{Quarto Capítulo}

\graphicspath{{Chapters/Chapter3/Chapter3Figs/PNG/}{Chapters/Chapter3/Chapter3Figs/PDF/}{Chapters/Chapter3/Chapter3Figs/}}


\section{Primeiro Parágrafo}

Segue-se mais uma equação:
\begin{equation}
a=\frac{N}{A}
\end{equation}%

\nomenclature{$a$}{The number of angels per unit area}%
\nomenclature{$N$}{The number of angels per needle point}%
\nomenclature{$A$}{The area of the needle point}%

The equation $\sigma = m a$%
\nomenclature{$\sigma$}{The total mass of angels per unit area}%
\nomenclature{$m$}{The mass of one angel}
follows easily.

\def\baselinestretch{1}
\chapter{Conclusões}

\graphicspath{{Chapters/Conclusions/ConclusionsFigs/PNG/}{Chapters/Conclusions/ConclusionsFigs/PDF/}{Chapters/Conclusions/ConclusionsFigs/}}



Segue-se as minhas conclusões...

\backmatter % book mode only
\appendix
\include{Chapters/Appendix/Appendix1/appendix1}
\include{Chapters/Appendix/Appendix2/appendix2}

%escolha um dos 3 estilos de bibliografia
\bibliographystyle{plainnat}

\renewcommand{\bibname}{Referências Bibliográficas}
\bibliography{References/references} 		% Caminho para o bibtex

\end{document}
